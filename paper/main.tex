\documentclass{article}

\PassOptionsToPackage{numbers, sort, compress}{natbib}
\usepackage[main,final]{neurips_2025}

\usepackage[T1]{fontenc}
\usepackage[utf8]{inputenc}
\usepackage{hyperref}
\usepackage{url}
\usepackage{booktabs}
\usepackage{nicefrac}
\usepackage{microtype}
\usepackage{xcolor}

%%%

\usepackage{subcaption}
\usepackage{graphicx}
\usepackage{multirow}
\usepackage{amsmath,amssymb,amsfonts}
\usepackage{amsthm}
\usepackage{mathrsfs}
\usepackage{xcolor}
\usepackage{textcomp}
\usepackage{manyfoot}
\usepackage{algorithm}
\usepackage{algorithmicx}
\usepackage{algpseudocode}
\usepackage{listings}

\newtheorem{theorem}{Theorem} % continuous numbers
%%\newtheorem{theorem}{Theorem}[section] % sectionwise numbers
%% optional argument [theorem] produces theorem numbering sequence instead of independent numbers for Proposition
\newtheorem{proposition}[theorem]{Proposition}% 
\newtheorem{lemma}{Lemma}% 
%%\newtheorem{proposition}{Proposition} % to get separate numbers for theorem and proposition etc.

\newtheorem{example}{Example}
\newtheorem{remark}{Remark}

\newtheorem{definition}{Definition}
\newtheorem{assumption}{Assumption}

%%%


\title{Title}


% The \author macro works with any number of authors. There are two commands
% used to separate the names and addresses of multiple authors: \And and \AND.
%
% Using \And between authors leaves it to LaTeX to determine where to break the
% lines. Using \AND forces a line break at that point. So, if LaTeX puts 3 of 4
% authors names on the first line, and the last on the second line, try using
% \AND instead of \And before the third author name.


\author{
  Name Surname\\
  MIPT\\
  Moscow, Russia\\
  \texttt{email@phystech.edu}\\
  \And
  Name Surname\\
  MIPT\\
  Moscow, Russia\\
  \texttt{email@phystech.edu}\\
}


\begin{document}


\maketitle

\begin{abstract}
    TODO
\end{abstract}

\textbf{Keywords:} TODO

\section{Introduction}\label{sec:intro}

TODO

\textbf{Contributions.} Our contributions can be summarized as follows:
\begin{itemize}
    \item We present...
    \item We demonstrate the validity of our theoretical results through empirical studies...
    \item We highlight the implications of our findings for...
\end{itemize}

\textbf{Outline.} The rest of the paper is organized as follows...

\section{Related Work}\label{sec:rw}

\textbf{Topic \#1.}
TODO

\textbf{Topic \#2.}
TODO

\section{Preliminaries}\label{sec:prelim}

\subsection{General notation}

In this section, we introduce the general notation used in the rest of the paper and the basic assumptions. 

\subsection{Assumptions} 

TODO

\section{Method}\label{sec:method}

\section{Experiments}\label{sec:exp}

To verify the theoretical estimates obtained, we conducted a detailed empirical study...

\section{Discussion}\label{sec:disc}

TODO

\section{Conclusion}\label{sec:concl}

TODO


%%%%%%%%%%%%%%%%%%%%%%%%%%%%%%%%%%%%%%%%%%%%%%%%%%%%%%%%%%%%

\bibliographystyle{unsrtnat}
\bibliography{references}

%%%%%%%%%%%%%%%%%%%%%%%%%%%%%%%%%%%%%%%%%%%%%%%%%%%%%%%%%%%%

\newpage
\appendix
\section{Appendix / supplemental material}\label{app}

\subsection{Additional experiments / Proofs of Theorems}\label{app:exp}

TODO

\end{document}