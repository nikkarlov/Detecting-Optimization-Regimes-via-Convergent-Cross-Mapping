\documentclass[aspectratio=169]{beamer}

% \graphicspath{{figs}}  % your figures here

\usepackage{cmap}
\usepackage{tikz}
\usepackage{array}
\usepackage{multicol}
\usepackage{booktabs}
\usepackage{csquotes}
\usepackage{amssymb,amsfonts,amsmath}

\usecolortheme{beaver}
\usefonttheme[onlymath]{serif}  % math fonts as in article
\setbeamertemplate{navigation symbols}{}
\setbeamertemplate{footline}[page number]
\setbeamertemplate{blocks}[rounded=true,shadow=true]
\setbeamersize{text margin left=0.5cm,text margin right=0.5cm}
\setbeamerfont{author}{size=\normalsize}
\setbeamerfont{institute}{size=\normalsize}
\setbeamerfont{date}{size=\normalsize}

\title{Title}
\author{
    Name Surname\\
    Consultant: Name Surname, PhD/DSc\\
    Expert: Name Surname, PhD/DSc
}
\institute{
    Course: My first scientific paper\\
    Intelligent Systems Department, MIPT
}
\date{Date}

\begin{document}

\begin{frame}
    \thispagestyle{empty}
    \maketitle
\end{frame}

\begin{frame}{Title or a main part of it}
    Couple of motivational sentences ...
    \begin{block}{Goal}
        Investigate ... 
    \end{block}
    \begin{block}{Problem}
        Put the brief idea here ...
    \end{block}
    \begin{block}{Solution}
        Your results appears twice, as a promise here and as a contribution later ...
        \begin{enumerate}[1)]
            \item Set ...
            \item Put ...
            \item Get ...
        \end{enumerate}
    \end{block}
\end{frame}

\begin{frame}{One-slide talk}
    What the audience sees on the figure.
    \begin{columns}
    \begin{column}{0.5\textwidth}
        \begin{enumerate}[1)]
            \item Notation
            \item Subjects
        \end{enumerate}
    \end{column}
        \begin{column}{0.5\textwidth}
            \begin{figure}[h]
                \centering
                \includegraphics[width=0.8\textwidth]{example-image}
                \caption{Caption}
            \end{figure}    
        \end{column}
    \end{columns}
    What are the consequences?
\end{frame}

\begin{frame}{Problem statement}
    Given a dataset $\mathfrak{D} = \left\{ (\mathbf{x}_i, y_i) \right\}_{i=1}^{N}$ ...
\end{frame}

\begin{frame}{Solution}
    We propose ...
\end{frame}

\begin{frame}{Experiments}
    We consider a MNIST dataset ...
    \begin{columns}
        \begin{column}{0.5\textwidth}
            \begin{figure}[h]
                \centering
                \includegraphics[width=0.7\textwidth]{example-image-a}
                \caption{Caption A}
            \end{figure}    
        \end{column}
        \begin{column}{0.5\textwidth}
            \begin{figure}[h]
                \centering
                \includegraphics[width=0.7\textwidth]{example-image-b}
                \caption{Caption B}
            \end{figure}    
        \end{column}
    \end{columns}
    Results show ...
\end{frame}

\begin{frame}{Conclusion}
    \begin{enumerate}[1)]
        \item We consider a problem ...
        \item We propose ...
        \item Experiments show ...
        \item Our method outperforms ...
    \end{enumerate}
\end{frame}

\end{document} 